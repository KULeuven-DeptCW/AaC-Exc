\documentclass[a4paper]{article}
\usepackage[dutch]{babel}
\usepackage{color,xypic,amsmath}
\xyoption{all}
\usepackage{../assignment-nl,../brackets,../importsreferences-nl}

\title{Automaten en Berekenbaarheid\\Opgave \#4\\\url{http://goo.gl/1RlVG5}}
\author{prof. B. Demoen\\W. Van Onsem}
\date{Oktober 2014}

\newcommand{\rul}{\rightarrow}
\newcommand{\gvar}[1]{\langle \text{{#1}} \rangle}
\newcommand{\gend}[1]{\text{\bf{#1}}}


\begin{document}

\maketitle

\begin{question}
Toon aan dat de klasse van context vrije talen gesloten is onder concatenatie en ster.

\begin{answer}
Meerdere manieren zijn mogelijk: we kunnen een constructie voorstellen voor een PDA of een context-vrije grammatica.
We opteren hier voor het laatste.
\begin{enumerate}
 \item Concatenatie
 \begin{construction}
  Gegeven twee context-vrije grammatica's $\tupl{V_1,\Sigma,R_1,S_1}$ en $\tupl{V_2,\Sigma,R_2,S_2}$. We maken de assumptie
  dat $V_1\cap V_2=\emptyset$: met andere woorden de beide context-vrije grammatica's bevatten geen gemeenschappelijke
  non-terminalen. Indien dit wel het geval is, volstaat het om de non-terminalen van de $V_1$ van \'e\'en van de context-vrije
  grammatica's te hernoemen (en deze substitutie ook toe te passen op de omzettingsregels $R_1$).
  
  Dan beschouwen we een contextvrije grammatica $\tupl{V,\Sigma,R,S}$ met:
  \begin{eqnarray}
   V&=&V_1\cup V_2\cup\accl{S}\\
   R&=&R_1\cup R_2\cup\accl{S\rightarrow S_1S_2}
  \end{eqnarray}
  Met $S$ een nieuwe ingevoerde non-terminaal.
 \end{construction}
 Deze constructie werkt omdat $S_1$ zal evolueren naar elke mogelijke string $s\in L_1$ en $S_2$ naar elke mogelijke
 string in $s\in L_2$. Omdat $S$ als enige overgangsregel de concatenatie van evoluties van $S_1$ en $S_2$ heeft, worden
 er geen bijkomende strings ge\"introduceerd.
 \item Kleene-ster
 \begin{construction}
  Gegeven een context-vrije grammatica $\tupl{V_1,\Sigma,R_1,S_1}$, dan construeren we een context-vrije grammatica $\tupl{V,\Sigma,R,S}$
  met:
  \begin{eqnarray}
   V&=&V\cup\accl{S}\\
   R&=&R_1\cup\accl{S\rightarrow\epsilon,S\rightarrow S_1S}
  \end{eqnarray}
  Met $S\notin V_1$.
 \end{construction}
 Omdat de Kleene ster impliceert dat we de originele taal $L_1$ nul of meer keer herhalen, beschouwen we eerst de overgangsregel $S\rightarrow\epsilon$ die
 impliceert dat de lege string deel uitmaakt van de taal: nul keer een string herhalen is dus mogelijk. $S_1$ kan overgaan in elke mogelijke string
 $s\in L$, wanneer we dus de overgangsregel $S\rightarrow S_1$ zouden beschouwen, maken alle strings in $L_1$ deel uit van $L$. Door $S$ toe te voegen aan (het
 einde van) de regel, kunnen we \'e\'en of meer concatenaties van elementen uit $L_1$ verzekeren.
\end{enumerate}
\end{answer}
\end{question}

\begin{question}
Bewijs dat de doorsnede van een reguliere taal met een context-vrije taal context-vrij is.
\end{question}

\begin{question}
Geef voor elk van de volgende talen een CFG die die taal genereert.
\begin{enumerate}
 \item $\{ w \in \{0,1\}^* \ | \ \text{de lengte van $w$ is oneven en het middelste symbool is een $0$} \}$
 \item $\{ w \in \{0,1\}^* \ | \ \text{$w$ bevat strikt meer $1$'en dan $0$'en} \}$
 \item $\{ w \in \{0,1\}^* \ | \ \text{$w$ bevat tweemaal zoveel $1$'en als $0$'en} \}$
 \item $\{ w_1 \# w_2 \# \ldots \# w_n \ | \ \text{$n \geq 1$, elke $w_i \in \{0,1\}^*$ en er bestaat een $i$ en een $j$ waarvoor $w_i = w_j^\mathcal{R}$} \}$
\end{enumerate}

\begin{answer}~~
\begin{enumerate}
 \item $\{ w \in \{0,1\}^* \ | \ \text{de lengte van $w$ is oneven en het middelste symbool is een $0$} \}$
 \importgram{exc4-1}{Oneven lengte, middelste symbool is $0$.}
 \item $\{ w \in \{0,1\}^* \ | \ \text{$w$ bevat strikt meer $1$'en dan $0$'en} \}$
 \importgram{exc4-2}{Meer $1$'en dan $0$'en.}
 \item $\{ w \in \{0,1\}^* \ | \ \text{$w$ bevat tweemaal zoveel $1$'en als $0$'en} \}$
 \importgram{exc4-3}{Tweemaal zoveel $1$'en als $0$'en.}
 \item $\{ w_1 \# w_2 \# \ldots \# w_n \ | \ \text{$n \geq 1$, elke $w_i \in \{0,1\}^*$ en er bestaat een $i$ en een $j$ waarvoor $w_i = w_j^\mathcal{R}$} \}$
 \importgram{exc4-4}{Omgekeerde groep.}
\end{enumerate}
\end{answer}
\end{question}

\begin{question}
Geef voor de eerste twee talen uit $(3)$ een PDA die die taal herkent.
\end{question}

\begin{question}
Zij $G$ een CFG in Chomsky normaalvorm die precies $b$ niet-terminalen bevat.
\begin{enumerate}
  \item Toon aan dat voor elke $w \in L_G$ met lengte $n \geq 1$, elke afleiding van $w$ uit precies $2n - 1$ stappen bestaat.
  \item Toon aan dat, als $L_G$ een string bevat waarvoor elke afleiding meer dan $2^b$ stappen bevat, $L_G$ oneindig moet zijn.
\end{enumerate}
\end{question}

\begin{question}
Toon aan dat $\{ x \# y \ | \ \text{$x,y \in \{0,1\}^*$ en $x \neq y$} \}$ een context-vrije taal is over $\{ 0,1,\# \}$.
\end{question}

\end{document}
