\documentclass[a4paper]{article}
\usepackage[dutch]{babel}
\usepackage{color,xypic,amsmath}
\xyoption{all}
\usepackage{../assignment-nl,../brackets}

\title{Automaten en Berekenbaarheid\\Opgave \#5\\\url{http://goo.gl/1RlVG5}}
\author{prof. B. Demoen\\W. Van Onsem}
\date{November 2014}

\newcommand{\rul}{\rightarrow}
\newcommand{\gvar}[1]{\langle \text{{#1}} \rangle}
\newcommand{\gend}[1]{\text{\bf{#1}}}


\begin{document}

\maketitle

\begin{question}
Welke van de volgende talen zijn context-vrij, welke niet? Bewijs je antwoord.
  \begin{enumerate}
		    \item $\{ w \in \{0,1\}^* \ | \ \text{$w$ is een palindroom}\}$
      \item $\{ 0^n\#0^{2n}\#0^{3n} \ | \ n \geq 0 \}$
      \item $\{ w \in \{a,b,c\}^* \ | \ \text{$w$ bevat een gelijk aantal $a$'s, $b$'s en $c$'s} \}$
      \item $\{ xy \ | \ \text{$x,y \in \{0,1\}^*$ en $|x| = |y|$, maar $x \neq y$} \}$
      \item $\{ w \# x \ | \ \text{$w,x \in \{0,1\}^*$ en $w$ is een substring van $x$} \}$
		    \item $\{a^ib^jc^k \ | \ 0 \leq i \leq j \leq k \}$
  \end{enumerate}
\begin{answer}
\begin{enumerate}~~
\item \textbf{contextvrij}: We kunnen deze taal beslissen met volgende grammatica.
\item \textbf{niet-contextvrij}: Dit kunnen we aantonen met het pompend lemma.
\begin{proof}
Gegeven een string $s=O^p\#O^{2p}\#O^{3p}$ met $p$ de onbekende pomplengte. Het is duidelijk dat $\abs{s}\geq p$. Voor elke opdeling $s=uvxyz$ die voldoet aan regels (2) en (3) geldt: dat $v$ en $y$ onmogelijk een $\#$ kunnen bevatten: immers zou dit betekenen dat we door te ``pompen'' meer van deze karakters in de string kunnen introduceren, wat niet kan volgens de taal. Het gevolg is dat ofwel (1) $v$ en $y$ tot \textbf{eenzelfde} groep $O$'s behoren, ofwel (2) tot twee opeenvolgende. In het eerste geval zullen we door te ``pompen'', meer $O$'s in deze groep introduceren (immers $\abs{vy}>0$), waardoor de boekhouding niet meer klopt. In het tweede geval kunnen we misschien de boekhouding van twee opeenvolgende groepen nog laten kloppen (bijvoorbeeld dat de tweede groep dubbel zoveel $O$'s als de eerste groep bevat), maar we kunnen enkel de karakters in twee groepen laten groeien, terwijl ze in alle drie de groepen volgens de gestelde wetmatigheid moeten groeien. Bijgevolg bestaat er geen geldige opdeling.
\end{proof}
\item \textbf{niet-contextvrij}: Dit kunnen we aantonen met het pompend lemma.
\begin{proof}
We beschouwen de string $s=a^pb^pc^p$ met $p$ de pomplengte. Wanneer we $s$ opdelen in $s=uvxyz$ zodat de twee laatste voorwaarden voldoen, betekent dit dat $v$ en $y$ ofwel uit strings van hetzelfde karakter bestaan (dus $v,y\in a^{\star}$), ofwel uit twee opeenvolgende karakters (bijvoorbeeld $u\in aa^{\star}$ en $y\in a^{\star}bb^{\star}$). Het is echter onmogelijk dat $v$ en $y$ strings zijn die bestaan uit alle drie de karakters. Bij het ``pompen'' zal echter minstens \'e\'en van de karakters in aantal toenemen. Omdat ze niet alle drie kunnen toenemen, is aan de voorwaarde van de taal niet meer voldaan. Dus inconsistentie.
\end{proof}
\item \textbf{context-vrij}: We kunnen deze taal beslissen met de PDA op \figref{}. De PDA werkt als volgt: eerst bouwen we de stack op voor $x$ op een willekeurig moment beschouwen we het begin van $y$\footnote{Omdat een PDA non-deterministisch is, mogen we gokken, immers moet er maar \'e\'en gok slagen om de taal te accepteren.}. Vervolgens bouwen we de stack terug af. We houden ondertussen bij van rechts naar links of de karakters rechts van de twee strings gelijk zijn of niet. Vanaf het moment dat dit voor \'e\'en karakter niet geldt, is dit het geval. We accepteren als de stack op het einde leeg is.
\item \textbf{context-vrij}: We kunnen deze taal beslissen met de PDA op \figref{}. De PDA werkt als volgt, eerst slaan we $w$ volledig op op de stack
\item \textbf{niet-contextvrij}
\end{enumerate}
\end{answer}
\end{question}

\begin{question}
Een $k$-PDA is een PDA met $k$ stacks. Een $0$-PDA is dus een NFA en een $1$-PDA een gewone PDA. 
\begin{enumerate}
  \item Toon aan dat $2$-PDA's strikt sterker zijn dan $1$-PDA's. (Met andere woorden, toon aan dat de klasse van talen die met een $1$-PDA te herkennen zijn een strikte deelverzameling is van de klasse van talen die met een $2$-PDA te herkennen zijn.)
  \item Toon aan dat $3$-PDA's even sterk zijn als $2$-PDA's.
\end{enumerate}
\end{question}
\begin{answer}
\begin{enumerate}
\item Dit bewijzen we in twee stappen: (a) we bewijzen dat elke taal die beslist kan worden door een $1$-PDA ook door een $2$-PDA beslist kan worden, en (b) we tonen aan dat er minstens \'e\'en taal bestaat die niet door een $1$-PDA beslist kan worden, maar wel door een $2$-PDA.
\begin{enumerate}
 \item Dit is triviaal waar: we kunnen immers elke $1$-PDA omvormen naar een $2$-PDA waar we de tweede stapel niet gebruiken (we voorzien dus telkens een ``no operation'' overgangsregel $\epsilon\rightarrow\epsilon$).
 \item Dit bewijzen we door aan te tonen dat de taal $\{ 0^n\#0^{2n}\#0^{3n} \ | \ n \geq 0 \}$ (zie eerder) wel beslisbaar is door een $2$-PDA. Dit staat beschreven op \figref{}. De PDA leest de eerste groep $O$'s in en zet op de eerste stapel overal $2$'en en op de tweede stapel $3$'en. Bij de tweede groep wordt de eerste stapel afgebouwd: we herschrijven een $2$ tot een $1$ en we halen $1$'en van de stack. Bij de derde groep bouwen we de tweede stack op een gelijkaardige manier af. Een $2$-PDA kan de taal dus wel beslissen.
\end{enumerate}
\item 
\begin{note}
Merk op dat een $2$-PDA even sterk is als een Turing machine en er dus ook mee samenvalt. We kunnen immers de eerste stack gebruiken om alle karakters links van de lees/schrijfkop voor te stellen (waarbij de ``top'' van de stack het dichtste karakter bij de kop bevat), en de tweede stack bevat de karakters rechts van de stack (opnieuw met het dichtste karakter als ``top''). De toestand kan het huidige karakter bijhouden (of we plaatsen het op \'e\'en van de stacks). Wanneer de kop naar links of rechts beweegt, kunnen we dan met wat boekhouding de stacks correct aanpassen. Er is dus geen rekeninstrument (behalve een orakel) die krachtiger is dan een $2$-PDA.
\end{note}
\end{enumerate}
\end{answer}
\begin{question}
Zij 
\begin{align*}
  \Sigma & = \{ \gend{if, condition, then, else, a:=1} \} \\
  V & = \{ \gvar{STMT}, \gvar{IF-THEN}, \gvar{IF-THEN-ELSE}, \gvar{ASSIGN} \} \\
\end{align*}
  en zij $G = (V,\Sigma,R,\gvar{STMT})$ de grammatica met regels
  \begin{align*}                                                                               
      \gvar{STMT}          & \rul \gvar{ASSIGN} \ | \ \gvar{IF-THEN} \ | \ \gvar{IF-THEN-ELSE}  \\
      \gvar{IF-THEN}       & \rul \gend{if condition then } \gvar{STMT} \\
      \gvar{IF-THEN-ELSE}  & \rul \gend{if condition then } \gvar{STMT} \gend{ else } \gvar{STMT} \\
      \gvar{ASSIGN}        & \rul \gend{a:=1} \\
  \end{align*}
  Toon aan dat $G$ ambigu is en construeer een niet-ambigue grammatica die dezelfde taal herkent.
\begin{answer}
Beschouw volgende expressie: \texttt{if condition then if condition then a := 1; else a := 1;}. Deze expressie is ambigu, omdat het niet duidelijk is tot welke \texttt{then}, de \texttt{else} behoort\footnote{Doordat de meeste talen gebruik maken van LALR parsing waarbij men geneigd is de \texttt{else} te binden aan de dichtste \texttt{then} (die nog niet gebonden is), kan het moeilijk zijn de ambiguiteit meteen te zien.}. Immers kan men dit groeperen (de haakjes tonen de oorsprong van de overgangsregels) als \texttt{(if condition then (if condition then (a := 1) else (a := 1)))} of als \texttt{(if condition then (if condition then (a := 1)) else (a := 1))}.
\end{answer}
\end{question}
\end{document}
