\documentclass[a4paper]{article}
\usepackage[dutch]{babel}
\usepackage{color,xypic,amsmath}
\xyoption{all}
\usepackage{../assignment-nl,../brackets}

\title{Automaten en Berekenbaarheid\\Opgave \#6\\\url{http://goo.gl/1RlVG5}}
\author{prof. B. Demoen\\W. Van Onsem}
\date{November 2014}


\begin{document}
\maketitle

\begin{question}
Een \emph{`blijf staan Turing machine'} verschilt van een gewone Turing machine doordat de head naar rechts kan bewegen of kan blijven staan, maar niet naar links kan bewegen. Toon aan dat deze variant van Turing machines niet equivalent is met de originele variant. Welke talen worden herkend door `blijf staan Turing machines'?
\begin{answer}
De verzameling van talen die beslist kan worden met zo'n machine is equivalent aan de verzameling van reguliere talen. We bewijzen de equivalentie in twee richtingen.
\begin{enumerate}
 \item Elke reguliere taal kan belist worden door een `blijf staan Turing machine':
\begin{proof}
Gegeven een reguliere taal $L$, dan bestaat er een DFA $\tupl{Q,\Sigma,\delta,q_s,F}$ die deze taal beslist. We kunnen aan de hand van volgende constructie een Turing machine bouwen die dezelfde taal beslist.
\begin{construction}
We construeren een Turing machine $\tupl{Q\cup\accl{q_a,q_f},\Sigma,\Sigma\cup\accl{\#},\delta',q_s,q_a,q_f}$ met:
\begin{eqnarray}
\forall q\in Q,\sigma\in\Sigma:\fun{\delta'}{q,\sigma}&=&\tupl{\fun{\delta}{q,\sigma},\#,R}\\
\forall q\in F,\sigma\in\Sigma:\fun{\delta'}{q,\#}&=&\tupl{q_a,\#,R}\\
\forall q\in Q\setminus F,\sigma\in\Sigma:\fun{\delta'}{q,\#}&=&\tupl{q_r,\#,R}
\end{eqnarray}
\end{construction}
De Turing machine redeneert dus over dezelfde verzameling toestanden en maakt dezelfde overgangen als de DFA. Verder gaat de machine telkens naar rechts. Wat er geschreven wordt op de tape is in principe irrelevant (hier $\#$, maar elk karakter $\gamma\in\Gamma$ is correct). Wanneer we het einde van de invoer bereiken, kijken we of we in een accepterende toestand van de DFA zaten. Indien zo nemen we de overgang naar de accept-toestand $q_a$, anders verwerpen we met $q_r$.
\end{proof}
 \item Elke `blijf staan Turing machine' beslist een reguliere taal:
\begin{proof}
Gegeven een `blijf staan Turing machine' $\tupl{Q,\Sigma,\Gamma,\delta,q_s,q_a,q_r}$. Dan zijn er twee soorten regels:
\begin{enumerate}
 \item regels waarbij de lees/schrijf-kop naar rechts beweegt. Omdat de lees/schrijfkop enkel naar rechts kan bewegen of blijft staan is wat we schrijven irrelevant.
 \item regels waarbij de lees/schrijf-kop blijft staan. In dat geval schrijven we een karakter weg, die onmiddellijk weer zal worden ingelezen door de volgende toestand.
Een DFA ``schuift'' bij elke invoer altijd onmiddellijk op naar rechts. We dienen dus eerst overgangsregels van de tweede soort te elimineren. Dit doen we door volgende reductie te blijven toepassen op de verzameling regels, tot er geen kandidaten meer gevonden kunnen worden:
\begin{quote}
Voor elke twee overgangsregels $1:\tupl{q_1,\gamma_1}\rightarrow\tupl{q_2,\gamma_2,S}$ en $2:\tupl{q_2,\gamma_2}\rightarrow\tupl{q_3,\gamma_3,m}$, vervang de regel door $\tupl{q_1,\gamma_1}\rightarrow\tupl{q_3,\gamma_3,m}$. $m$ kan hier ofwel $S$ ofwel $R$ zijn. Behouw de tweede regel tot er geen kandidaten meer voor de eerste regel bestaan.
\end{quote}
\end{enumerate}
Het resultaat van deze omzetting herhaaldelijk toe te passen is dat de Turing machine enkel nog naar rechts beweegt. We hebben de overgangen waar de Turing machine bleef staan, ``kortgesloten''. Het is niet helemaal uitgesloten dat er niet nog een blijf staan instructies in de Turing machine staat: dit is het gevolg van een lus van van blijf staan instructies in de originele Turing machine. In dat geval komt de Turing machine dus in een oneindige lus terecht, en horen we de string te verwerpen.
\paragraph{}
We modificeren de Turing machine verder door toestanden die niet bereikbaar zijn vanuit $q_s$ te verwijderen. Nu dienen we de gemodificeerde ``blijf staan Turing machine'' nog om te zetten naar een DFA. Dit doen we met volgende constructie.
\begin{construction}
Gegeven de gemodificeerde ``blijf staan Turing machine'' $\tupl{Q,\Sigma,\Gamma,\delta,q_s,q_a,q_r}$, we construeren een DFA $\tupl{Q,\Sigma,\delta',q_s,F}$ als volgt:
\begin{eqnarray}
F&=&\accl{q_a}\\
\forall q\in Q\setminus\accl{q_a,q_r},\sigma\in\Sigma:\fun{\delta'}{q,\sigma}&=&\acclguard{q_r&\xif m=S\\q'&\xotherwise}\xwhere\tupl{q',\sigma',m}=\fun{\delta}{q,\sigma}\\
\forall q\in\accl{q_a,q_r}\forall\sigma\in\Sigma:\fun{\delta'}{q,\sigma}&=&q
\end{eqnarray}
\end{construction}
We beschouwen dus dezelfde verzameling toestanden. Maar de $q_r$ en $q_a$ toestand, zijn toestanden waarin de DFA blijft zitten tot alle karakters zijn opgebruikt. Enkel in het geval van $q_a$ accepteren we ook. Voor de overige toestanden volgen we de overgangsfunctie van de gemodificeerde Turing machine. Indien deze machine echter blijft stilstaan, geraken we in een oneindige lus. Dit lossen we op door naar deze strings af te leiden naar $q_r$.
\end{proof}
\end{enumerate}
\end{answer}
\end{question}

\begin{question}
Zij $INF_{\rm DFA} = \{ \langle A \rangle \ | \ \text{$A$ is een DFA en $L(A)$ is een oneindige taal} \} $. Toon aan dat $INF_{\rm DFA}$ beslisbaar is.
\begin{answer}
\begin{lemma}
Een DFA belist een oneindige taal als en slechts als ze een lus bevat \textbf{en} paden vanuit die lus in een accepterende toestand kunnen bereiken.
\begin{proof}
Een DFA beslist een oneindige taal indien oneindig veel paden in een accepterende toestand terecht komen\footnote{Immers komt elk pad overeen met een unieke string}. Indien de DFA een lus bevat, betekent dat oneindig veel paden en dus ook strings toekomen in de toestanden die de lus aandoet: immers kunnen we de lus een willekeurig aantal keer herhalen. Indien een toestand die in de lus vervat zit tot een accepterende toestand leidt (of zelf een accepterende toestand is), betekent dit dat we door een vast aantal karakters toe te voegen aan elke string van elke pad, we een accepterende string genereren. Omdat het aantal paden oneindig is, is het aantal strings die geaccepteerd worden, ook oneindig.
\end{proof}
\end{lemma}
Op basis van dit lemma kunnen we gegeven een DFA bepalen of de taal die de DFA beslist oneindig is: we voeren eerst een lusdetectie algoritme uit. Vervolgens voeren we een bereikbaarheidsanalyse uit op de lussen. Indien minstens \'e\'en lus nog een accepterende toestand bereikt . Het beschreven algoritme werkt in \bigoh{n^4}, met $n$ het aantal toestanden. Een intelligenter algoritme werkt echter in \bigoh{n^2} wat een bewezen optimale tijdscomplexiteit is.
\end{answer}
\end{question}

\begin{question}
Is $VIJF_{\rm DFA} = \{ \langle A \rangle \ | \ \text{$A$ is een DFA en $L(A)$ bestaat uit precies 5 strings} \} $ beslisbaar? Bewijs je antwoord.
\begin{answer}
Ja.
\begin{proof}
We kunnen eerst de vorige oefening gebruiken om te controleren of de DFA een eindige taal beslist. Indien de DFA $n$ toestanden bevat en geen oneindige taal beslist, kunnen de strings hoogstens een lengte $n$ hebben, immers zouden we om een string met een lengte groter dan $n$ te accepteren, gebruik moeten maken van een lus om in een accepterende toestand terecht te komen, waardoor de taal oneindig is. We kunnen vervolgens alle strings tot en met lengte $n$ opsommen en op de DFA laten lopen. Wanneer er precies $5$ zo'n strings worden geaccepteerd, accepteren we. Anders rejecten we. Het algoritme werkt dus in \bigoh{n^4+\abs{\Sigma}^n}.
\end{proof}
\begin{note}
Een effici\"enter maar ook complexer algoritme maakt gebruik van DFA minimalisatie.
\end{note}
\end{answer}
\end{question}

\begin{question}
Toon aan dat de vraag of een context-vrije grammatica minstens \'e\'en string uit $1^*$ kan genereren, beslisbaar is.
\begin{answer}
De doorsnede van een context-vrije taal met een reguliere taal is een context-vrije taal. We hebben hiervoor een constructie gezien in een vorige oefenzitting. Men kan dus een PDA construeren die de doorsnede beslist. We vormen vervolgens die PDA om tot een equivalente context-vrije grammatica. Stelling $10.5$ beschrijft vervolgens op hoog niveau een algoritme die bepaalt of een gegeven context-vrije grammatica een lege taal beslist. Indien de resulterende taal leeg is, verwerpen we de grammatica, anders aanvaarden we de grammatica.
\end{answer}
\end{question}

\begin{question}
Zij $A$ en $B$ twee disjuncte talen die co-Turing-herkenbaar zijn. Toon aan dat er een beslisbare taal $C$ bestaat zodanig dat $A \subseteq C$ en $B \subseteq \overline{C}$.
\begin{answer}
We introduceren eerst het concept van ``parallel lopen'', dit is een concept die in heel wat beslissingsvraagstukken een rol kan spelen. Een universele Turing machine is een Turing machine die de beschrijving van een Turing machine inleest en deze vervolgens kan uitvoeren op een string. Met andere woorden een programma die een programma uitvoert. Het is een kleine stap om een universele Turing machine om te bouwen tot een machine die de beschrijving van twee Turing machines als invoer neemt, en vervolgens vervlochten\footnote{Engels: interleaved.} eerst een stap op de ene Turing machine uitvoert, en daarna een stap op de andere. We nemen aan dat dit kan.
\paragraph{}
Met deze tool kunnen we een Turing machine bouwen voor $C$, de Turing machine laat de co-herkenners voor $A$ en $B$ ``parallel'' lopen. Als de Turing machine voor $A$ accepteert, verwerpen we, als de co-herkenner voor $B$ accepteert, accepteren we de string. Een string die door beide machine geaccepteerd zou worden, wordt dus beslist op basis van de machine die het eerste accepteert\footnote{En indien dit in dezelfde stap gebeurd, spreken we een tie-breaker af, bijvoorbeeld dat co-herkenner voor $A$ prioriteit krijgt. Dit is echter volledig arbitrair: elke tie-breaker is te verdedigen.}.
\paragraph{}
Het is duidelijk dat de taal $C$ aan de voorwaarden voldoet: uit $A\subseteq C$ volgt dat indien een string $s$ niet in $C$ zit, deze ook niet in $A$ zit. De enige mogelijkheid waarom onze beslisser een string verwerpt is echter omdat de co-herkenner voor $A$ accepteert. De voorwaarde is dus rechtstreeks voldaan. Analoog betekent $B\subseteq\overline{C}$ dat indien indien onze beslisser een string aanvaard, deze niet tot $B$ mag behoren, wat ook rechtstreeks terug te vinden is in de beschrijving van ons algoritme. De specificaties in de vraag scheppen dus geen voorwaarden wat men moet doen wanneer beide co-herkenners (tegelijk) aantonen dat een string niet tot de taal behoort.
\paragraph{}
We dienen ook nog aan te tonen dat de machine een beslisser is: het algoritme moet altijd stoppen. Dit volgt uit de disjunctie van $A$ en $B$. Immers volgt uit $A\cap B=\emptyset$ dat $\overline{A\cap B}=\Sigma^{\star}$ en $\overline{A}\cup\overline{B}=\Sigma^{\star}$. Met andere woorden, voor elke string zal minstens \'e\'en van de co-herkenners ooit aantonen dat de string niet in de taal zit.
\end{answer}
\end{question}

\begin{question}
Een \emph{Turing machine met RESET} verschilt van een gewone Turing machine doordat de head enkel naar rechts kan bewegen of in \'e\'en stap helemaal aan het begin van de tape gezet kan worden. De transitiefunctie van zo een machine is dus van de vorm $\delta : Q \times \Gamma \to Q \times \Gamma \times \{ \text{R}, \text{RESET} \}$. Toon aan dat deze variant van Turing machines equivalent is met de originele variant.
\begin{answer}
We bewijzen in twee richtingen:
\begin{enumerate}
\item Een ``originele Turing machine'' is even krachtig als een ``Turing machine met RESET''.
\begin{proof}
We kunnen vanuit een ``Turing machine met RESET'' $T_1$ een ``originele Turing machine'' $T_2$ constureren door het tape-alfabet $\Gamma$ te dupliceren $\Gamma'=\Gamma\cup\condset{\gamma'}{\gamma\in\Gamma}$. De Turing machine markeert vervolgens eerst de beginpositie van de tape door het karakter die daar staat $\gamma$ te vervangen door $\gamma'$. Wanneer $T_1$ naar rechts beweegt doen we dit ook op $T_2$, we behandelen de karakters met of zonder marker allemaal alsof het karakters zijn zonder marker. Wanneer we karakters wegschrijven doen we dit zonder marker (behalve op de beginpositie). Indien $T_1$ de RESET-functie oproept, komen we in een speciale toestand terecht die terugkeert op de band tot het het karakter op de band vindt die gemarkeerd is. Vervolgens voert het de beschreven overgang uit, maar markeert ook het uitvoer-symbool (zodat we kunnen blijven bijhouden waar de tape begint).
\end{proof}
\item Een ``Turing machine met RESET'' is even krachtig als een ``originele Turing machine''.
\begin{proof}
Een ``Turing machine met RESET'' heeft slechts een oneindige band in \'e\'en richting: immers kunnen we hoogstens terugkeren naar het beginpunt, we kunnen dus de kop onmogelijk links van dit punt bewegen. Dit kunnen we oplossen door de band ``samen te plooien'': daar de band origineel bestond uit de karakters:
\begin{equation}
\ldots,l_3,l_2,l_1,s,r_1,r_2,r_3,\ldots
\end{equation}
Beschouwen we nu een tape met eerst het start-element $s$ en vervolgens afwisselend een element naar links en naar rechts:
\begin{equation}
s,l_1,r_1,l_2,r_2,l_3,r_3,\ldots
\end{equation}
Het vraagt enige extra boekhouding on een Turing machine aan te passen (men dupliceert bijvoorbeeld de toestanden: een toestand die stelt dat de Turing machine op het ``linkse gedeelte'' van de tape zit een een toestand voor het ``rechtse'' gedeelte), alsook dient men toestanden te voorzien zodat men sprongen van twee naar links en naar rechts kan maken, maar men kan een compiler bouwen die een Turing machine volgens dit paradigma laat werken. We hebben dus aangetoond dat een Turing machine die werkt met een tape oneindig in twee richtingen precies even sterk is als een Turing machine met een tape oneindig in juist \'e\'en richting.

Vervolgens beschrijven we informeel hoe we de bewegingen naar links en blijven staan op deze Turing machine, kunnen emuleren op een ``Turing machine met RESET''. Dit is gebaseerd op een verdrievoudiging van het tape-alfabet $\Gamma$. De eerste reeks zijn de originele karakters, de tweede reeks de karakters met een marker voor de positie van de lees/schrijfkop en de derde reeks zijn gok-markers.
\paragraph{}
De hoofdmarker beschrijft waar de kop zich bevindt op de tape. Indien de tape dus het karakter $a$ bevat, en het hoofd is daaronder gevestigd, dient er voortaan $\hat{a}$ te staan. Wanneer de Turing machine dus naar rechts beweegt, dient hij een karakter weg te schrijven zonder marker. De kop is er immers al gepasseerd.
\paragraph{}
Wanneer de originele Turing machine blijft staan, wordt dit door de ``Turing machine met RESET'' ge\"emuleerd door een karakter met marker weg te schrijven onder de tape en vervolgens een RESET beweging uit te voeren. Vervolgens beweegt men de kop weer naar rechts tot men het symbool met de marker tegenkomt waarna men verder gaat met het originele programma. Merk op dat men ook de toestand moet bijhouden waarin men zich bevond voor de RESET. Dit kan men doen door per toestand een ``specifieke RESET procedure'' te ontwikkelen, of bijvoorbeeld het tape-alfabet nog verder uit te breiden zodat men voor de RESET ook de toestand weg schrijft.
\paragraph{}
In het geval de Turing machine naar links gaat, kunnen we de hierboven beschreven procedure niet gebruiken, immers is het ``al te laat'' wanneer we het marker karakter opmerken. We kunnen dit oplossen aan de hand van ``gok-markers''. Vlak na de RESET gokken we dat de tape zich op de start-locatie zal bevinden. We zetten dus de gok-marker en voeren een reset uit. Vervolgens schuiven we naar rechts tot we de gok-marker tegenkomen (in dit geval dus na $0$ stappen), verwijderen de gok-marker, schuiven een vakje op naar rechts en kijken of rechts van de gok-marker inderdaad de echte marker stond. Zo ja, verwijderen we de echte marker, voeren een RESET uit, lopen terug naar de gok-marker en vervolgen vanaf daar het oude programma.
\paragraph{}
Wanneer rechts na het vakje van de gok-marker de echte marker niet staat, hebben we ons vergist. We hebben echter de gok-marker al verwijderd, en plaatsen de gok-marker juist \'e\'en vakje meer naar rechts. We doen dus een gok dat lees/schrijfkop dan maar een positie verder naar rechts ligt. We voeren opnieuw een RESET uit en controleren onze gok. We schuiven dus de gok-marker verder op tot we uiteindelijk ooit eens juist zullen gokken en we dus de marker inderdaad naar links hebben geplaatst. Deze operatie is echter zeer arbeidsintensief en komt neer op een kwadratische tijdscomplexiteit in de lengte van de tape op dat moment.
\end{proof}
\end{enumerate}
\end{answer}
\end{question}
\end{document}
