\documentclass[a4paper]{article}
\usepackage[dutch]{babel}
\usepackage{color,xypic,amsmath}
\xyoption{all}
\usepackage{../assignment-nl,../brackets}

\newcommand{\prnul}[0]{\ensuremath{\mbox{nul}}}
\newcommand{\prsucc}[0]{\ensuremath{\mbox{succ}}}
\newcommand{\prp}[2]{\ensuremath{p_{#2}^{#1}}}
\newcommand{\prcn}[2]{\funmf{Cn}{#1,#2}}
\newcommand{\prpr}[2]{\funmf{Pr}{#1,#2}}

\title{Automaten en Berekenbaarheid\\Opgave \#8\\\url{http://goo.gl/1RlVG5}}
\author{prof. B. Demoen\\W. Van Onsem}
\date{November 2014}

\newcommand{\N}{\mathbb{N}}

\begin{document}
\maketitle

\begin{question}
Zijn de volgende uitspraken waar of niet? Bewijs je antwoord.
\begin{enumerate}
  \item $EQ_{\rm CFG}$ is co-herkenbaar.
  \item $A_{\rm TM}$ kan gereduceerd worden naar $E_{\rm TM}$.
  \item Als een taal $L_1$ Turing gereduceerd kan worden naar een reguliere taal $L_2$, dan is $L_1$ ook regulier.
  \item Een taal is herkenbaar als en slechts als ze gereduceerd kan worden naar $A_{\rm TM}$.
\end{enumerate}
\begin{answer}~~
\begin{enumerate}
 \item Waar: we kunnen immers over alle mogelijke strings itereren, en telkens beslissen of een string $s$ inderdaad tot de gegeven context-vrije grammatica's behoort. Indien dit voor juist \'e\'en contextvrije grammatica het geval is, rejecten we de twee contextvrije grammatica's.
 \item 
 \item Niet waar, we kunnen immers elke Turing complete taal $L_1$ reduceren naar een reguliere taal $L_2=\accl{1}$: we laten eerst een Turing machine $M_{L_1}$ lopen op de invoer. Indien deze machine accepteert schrijven we $1$ op de tape en roepen we het orakel op, indien de machine reject, schrijven we bijvoorbeeld $0$ naar de tape.
 \item 
\end{enumerate}
\end{answer}
\end{question}

\begin{question}
Schrijf de volgende primitief recursieve functies met behulp van compositie, primitieve recursie en de basisfuncties.
\begin{enumerate}
  \item $exp : \N \times \N \to \N : (x,y) \mapsto x^y$
  \item $pred : \N \to \N : x \mapsto x - 1$ als $x \neq 0$ en $0 \mapsto 0$.
  \item $dif : \N \times \N \to \N : (x,y) \mapsto | x - y |$
\end{enumerate}
\begin{answer}~~
\begin{enumerate}
 \item $\prpr{\prcn{\prsucc}{\prnul}}{\prcn{\prpr{\prnul}{\prcn{\prpr{\prp{1}{1}}{\prcn{\prsucc}{\prp{3}{3}}}}{\prp{3}{1},\prp{3}{3}}}}{\prp{3}{1},\prp{3}{3}}}$
 \item $\prpr{\prnul}{\prp{3}{2}}$
\end{enumerate}
\end{answer}
\end{question}

\end{document}
