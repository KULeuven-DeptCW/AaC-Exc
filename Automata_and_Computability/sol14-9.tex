\documentclass[a4paper]{article}
\usepackage[dutch]{babel}
\usepackage{color,xypic,amsmath}
\xyoption{all}
\usepackage{../assignment-nl,../brackets}

\title{Automaten en Berekenbaarheid\\Opgave \#9\\\url{http://goo.gl/1RlVG5}}
\author{prof. B. Demoen\\W. Van Onsem}
\date{December 2014}

\newcommand{\twar}[2]{\left[ { }^{#1}_{#2} \right] }
\newcommand{\N}{\mathbb{N}}

\begin{document}
\maketitle

\begin{question}
Waar of niet waar?
\begin{enumerate}
  \item Elke eindige taal is regulier.
  \item Elke deelverzameling van een niet-reguliere taal is niet-regulier.
  \item De klasse van beslisbare talen is gesloten onder het nemen van het complement.
\end{enumerate}
\begin{answer}~~
\begin{enumerate}
 \item \textbf{Waar}: stel dat de taal $L={s_1,s_2,\ldots,s_n}$, kunnen we deze taal beslissen aan de hand van de reguliere expressie $s_1|s_2|\ldots|s_n$.
 \item \textbf{Niet waar}: een triviaal reguliere deelverzameling is de lege of elke eindige deelverzameling; dit volgt uit de vorige oefening.
 \item \textbf{Waar}.
 \begin{proof}
  Gegeven een beslisbare taal $L$, dan bestaat er een beslisser $B$ die de taal beslist. In dat geval is $\overline{L}$ ook beslisbaar: we kunnen immers een beslisser $\overline{B}$ construeren die $\overline{L}$ beslist: we laten eerst $B$ op de invoer lopen, wanneer $B$ accepteert verwerpen we en vice versa.
 \end{proof}
\end{enumerate}
\end{answer}
\end{question}

\begin{question}
Zij $L$ een reguliere taal over een alfabet $\Sigma$ met minstens 2 tekens. Definieer de afstand $d$ tussen twee strings $s$ en $t$ over van gelijke lengte als het aantal posities waarop $s$ en $t$ verschillen. Met andere woorden, als $s = s_1s_2\ldots s_n$ en $t = t_1t_2\ldots t_n$, dan is $d(s,t) = \#\{ i \mid s_i \neq t_i \}$. Definieer nu de taal
\[ fout_1(L) = \{s \in \Sigma^* \mid \text{er bestaat een $t \in L$ zodat $|s| == |t|$ en $d(s,t) \leq 1$} \}\]
Informeel is dat de taal met strings met hoogstens \'e\'en foutje ten opzichte van een string in L. Bewijs dat $fout_1(L)$ regulier is.
\begin{answer}
\begin{proof}
Dit kunnen we aantonen aan de hand van een constructie.
\begin{construction}
Gegeven een DFA $\tupl{Q,\Sigma,\delta,q_s,F}$ due $L$ beslist, dan construeren we een NFA $\tupl{Q',\Sigma,\delta',q_s',F'}$ met:
\begin{eqnarray}
Q'&=&Q\times\accl{0,1}\\
q_s'&=&\tupl{q_s,0}\\
F'&=&F\times\accl{0,1}\\
\forall q\in Q,\sigma\in\Sigma:\fun{\delta'}{\tupl{q,0},\sigma}&=&\condset{\tupl{q',0}}{q'=\fun{\delta}{q,\sigma}}\cup\condset{\tupl{q'',1}}{\forall \sigma'\in\Sigma:q''=\fun{\delta}{q,\sigma'}}\\
\forall q\in Q,\sigma\in\Sigma:\fun{\delta'}{\tupl{q,1},\sigma}&=&\accl{\tupl{q',1}}\xwhere q'=\fun{\delta}{q,\sigma}
\end{eqnarray}
\end{construction}
We dupliceren dus de toestanden en defini\"eren een niet-deterministische overgangsfunctie $\delta'$. Deze overgangsfunctie volgt grotendeels de originele overgangsfunctie. Wanneer we tot dusver nog geen fout zijn tegengekomen -- dit zijn toestanden van het type $\tupl{q,0}$ -- hebben we de keuze om zowel $\tupl{q',0}$ en $\tupl{q'',1}$ te nemen: we weten immers niet of $q'$ een ``vuilbaktoestand'' is. We dienen dus een ontsnappingsroute te voorzien naar elke toestand die bereikbaar is vanuit $q$ (het karakter maakt niet uit). In dat laatste geval wedden we dus op ``alle'' karakters tegelijk. Eenmaal we in een toestand $\tupl{q,1}$ terecht komen, volgen we opnieuw de originele overgangsregels. Omdat we geen tweede ontsnapping aanbieden, zal een NFA bij \'e\'en fout, dus zuinig met dit alternatief omspringen: immers moet er maar \'e\'en pad slagen, namelijk datgene waar de ``joker'' op de juiste plaats werd ingezet.
\end{proof}
\end{answer}
\end{question}

\begin{question}
Geef voor elk van de volgende talen aan of ze regulier, context-vrij, beslisbaar, Turing-herkenbaar en/of co-Turing-herkenbaar zijn.
\begin{enumerate}
  \item $\{ x = y+z \  | \ \text{$x$, $y$ en $z$ zijn natuurlijke getallen in binaire notatie en $x$ is de som van $y$ en $z$.} \}$ over het alfabet $\{0,1,+,=\}$.
  \item De taal van alle strings $w$ over het alfabet $\left\{ \twar{0}{0}, \twar{0}{1}, \twar{1}{0}, \twar{1}{1} \right\}$ zodanig dat het binair getal gevormd door de onderste rij van $w$ gelijk is aan drie maal het binair getal gevormd door de bovenste rij van $w$. 
  \item $\{ \langle M \rangle \ | \ \text{$M$ is een TM zodanig dat $L(M)$ precies 1234 strings bevat} \}$ % vraag: is deze taal co-Turing-herkenbaar?
  \item $\{ a^ib^jc^k \ | \ \text{$i,j,k \geq 0$ en $i \neq j$ of $j \neq k$} \}$ 
\end{enumerate}
\begin{answer}~~
\defaulttable{c|ccccc}{Taal&Regulier&Context-vrij&Beslisbaar&Herkenbaar&Co-herkenbaar}{
(a)&         &         &$\bullet$&$\bullet$&$\bullet$\\
(b)&$\bullet$&$\bullet$&$\bullet$&$\bullet$&$\bullet$\\
(c)&         &         &         &         &         \\
(d)&         &$\bullet$&$\bullet$&$\bullet$&$\bullet$
}
\end{answer}
\end{question}

\begin{question}
Defineer in zuivere lambda calculus de functie $sg$ zodanig dat $sg(c_n) = TRUE$ als $n = 0$ en $sg(c_n) = FALSE$ als $n \neq 0$.
\begin{answer}

\end{answer}
\end{question}

\begin{question}
Schrijf de volgende primitief recursieve functies met behulp van compositie, primitieve recursie en de basisfuncties.
\begin{enumerate}
  \item $\mbox{sg}:\NNN\to\NNN:x\mapsto1\text{ als $x = 0$ en $x \mapsto 0$ anders}$
  \item $max:\mathbb{N} \times \mathbb{N}\to\NNN:\tupl{a,b}\mapsto\text{ het maximum van $a$ en $b$}$
\end{enumerate}
\begin{answer}

\end{answer}
\end{question}
\end{document}